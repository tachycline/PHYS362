\documentclass[addpoints]{exam}
\usepackage{charter}

\begin{document}
\addpoints
\header{Physics 362}{Final Exam}{10 December 2010}
\coverchead{\Large Physics 362 Final Exam}
\cellwidth{4em}
\gradetablestretch{3.0}
\headrule

\begin{coverpages}
\vskip0.1cm\hrule\vskip0.5cm
\makebox[\textwidth]{Name:\enspace\hfill}
\vskip 1cm
You may use both your textbook and your homework portfolio as references with this exam.
\vskip 2cm
\begin{center}
\gradetable
\end{center}
\end{coverpages}

\begin{questions}
  \question {\bf A One dimensional potential.} Consider the potential
  \begin{equation} V = \left\{\begin{array}{r}\infty \qquad x\le 0\\
        -\alpha/x \qquad x>0 \end{array}\right. .\end{equation}
  \begin{parts}
    \part[8] Do you expect to find bound states for this potential?  What about scattering states?  Explain your answer\vspace{\fill}
    \part[8] How does this potential differ from the Coulomb potential (the potential found in the hydrogen atom)?\vspace{\fill}
    \part[8] What do you expect this difference will do to the eigenfunctions?  The energies?\vspace{\fill}
  \end{parts}\newpage
  \question {\bf A two state system.} The Hamiltonian for a cerain two-level system is 
\begin{equation}
  \hat H = -|1\rangle\langle 1| + \alpha |2\rangle\langle 1| -\alpha |1\rangle\langle 2| + |2\rangle\langle 2|
\end{equation}
  \begin{parts}
    \part[10] Find the eigenvalues of $\hat H$. \vspace{\fill}
    \part[10] Find the eigenfunctions of $\hat H$. \vspace{\fill}
    \part[8]  What is the matrix representing $\hat H$ in this basis?\vspace{\fill}
  \end{parts}\newpage

  \question {\bf 3D SHO, cartesian coordinates.} In three dimensions,
  the harmonic oscillator potential can be written
  \begin{equation} V = {m\omega^2\over 2}\left(x^2+y^2+z^2\right).
  \end{equation}
  \begin{parts}
    \part[8] Separate Schr\"odinger's equation into three ODEs for this
    potential.\newpage
    \part[8] Use work either in your portfolio or textbook to write down
    the energy eigenvalues ({\sl i.e.}, the allowed energies) and the
    eigenfunctions for this solution.\vspace{\fill}
    \part[8] Describe the degeneracy of these solutions.  How many states
    share the same energy, and where does the degeneracy come from,
    physically?\vspace{\fill}
  \end{parts}\newpage
  \question {\bf 3D SHO, spherical coordinates.} In three dimensions,
  the harmonic oscillator potential can also be written
  \begin{equation} V = {m \omega^2\over 2}r^2. \end{equation}
  \begin{parts}
    \part[8] Separate Schr\"odinger's equation into three ODEs for this
    potential. (Use spherical coordinates).\newpage
    \part[8] Do you expect the same degree of degeneracy in the solutions
    in spherical coordinates as you observed in cartesian coordinates?
    Why or why not?\vspace{\fill}
    \part[8] Give one reason you might choose spherical coordinates over
    cartesian coordinates for the three dimensional harmonic
    oscillator. Also give one reason you might choose cartesian
    coordinates over spherical coordinates for this problem.\vspace{\fill}
  \end{parts}
\end{questions}

\end{document}